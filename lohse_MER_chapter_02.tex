\documentclass[]{article}
\usepackage{lmodern}
\usepackage{amssymb,amsmath}
\usepackage{ifxetex,ifluatex}
\usepackage{fixltx2e} % provides \textsubscript
\ifnum 0\ifxetex 1\fi\ifluatex 1\fi=0 % if pdftex
  \usepackage[T1]{fontenc}
  \usepackage[utf8]{inputenc}
\else % if luatex or xelatex
  \ifxetex
    \usepackage{mathspec}
  \else
    \usepackage{fontspec}
  \fi
  \defaultfontfeatures{Ligatures=TeX,Scale=MatchLowercase}
\fi
% use upquote if available, for straight quotes in verbatim environments
\IfFileExists{upquote.sty}{\usepackage{upquote}}{}
% use microtype if available
\IfFileExists{microtype.sty}{%
\usepackage{microtype}
\UseMicrotypeSet[protrusion]{basicmath} % disable protrusion for tt fonts
}{}
\usepackage[margin=1in]{geometry}
\usepackage{hyperref}
\hypersetup{unicode=true,
            pdftitle={Chapter 2: Mixed-Effects Models for Factorial Designs},
            pdfauthor={Keith Lohse, PhD, PStat},
            pdfborder={0 0 0},
            breaklinks=true}
\urlstyle{same}  % don't use monospace font for urls
\usepackage{color}
\usepackage{fancyvrb}
\newcommand{\VerbBar}{|}
\newcommand{\VERB}{\Verb[commandchars=\\\{\}]}
\DefineVerbatimEnvironment{Highlighting}{Verbatim}{commandchars=\\\{\}}
% Add ',fontsize=\small' for more characters per line
\usepackage{framed}
\definecolor{shadecolor}{RGB}{248,248,248}
\newenvironment{Shaded}{\begin{snugshade}}{\end{snugshade}}
\newcommand{\AlertTok}[1]{\textcolor[rgb]{0.94,0.16,0.16}{#1}}
\newcommand{\AnnotationTok}[1]{\textcolor[rgb]{0.56,0.35,0.01}{\textbf{\textit{#1}}}}
\newcommand{\AttributeTok}[1]{\textcolor[rgb]{0.77,0.63,0.00}{#1}}
\newcommand{\BaseNTok}[1]{\textcolor[rgb]{0.00,0.00,0.81}{#1}}
\newcommand{\BuiltInTok}[1]{#1}
\newcommand{\CharTok}[1]{\textcolor[rgb]{0.31,0.60,0.02}{#1}}
\newcommand{\CommentTok}[1]{\textcolor[rgb]{0.56,0.35,0.01}{\textit{#1}}}
\newcommand{\CommentVarTok}[1]{\textcolor[rgb]{0.56,0.35,0.01}{\textbf{\textit{#1}}}}
\newcommand{\ConstantTok}[1]{\textcolor[rgb]{0.00,0.00,0.00}{#1}}
\newcommand{\ControlFlowTok}[1]{\textcolor[rgb]{0.13,0.29,0.53}{\textbf{#1}}}
\newcommand{\DataTypeTok}[1]{\textcolor[rgb]{0.13,0.29,0.53}{#1}}
\newcommand{\DecValTok}[1]{\textcolor[rgb]{0.00,0.00,0.81}{#1}}
\newcommand{\DocumentationTok}[1]{\textcolor[rgb]{0.56,0.35,0.01}{\textbf{\textit{#1}}}}
\newcommand{\ErrorTok}[1]{\textcolor[rgb]{0.64,0.00,0.00}{\textbf{#1}}}
\newcommand{\ExtensionTok}[1]{#1}
\newcommand{\FloatTok}[1]{\textcolor[rgb]{0.00,0.00,0.81}{#1}}
\newcommand{\FunctionTok}[1]{\textcolor[rgb]{0.00,0.00,0.00}{#1}}
\newcommand{\ImportTok}[1]{#1}
\newcommand{\InformationTok}[1]{\textcolor[rgb]{0.56,0.35,0.01}{\textbf{\textit{#1}}}}
\newcommand{\KeywordTok}[1]{\textcolor[rgb]{0.13,0.29,0.53}{\textbf{#1}}}
\newcommand{\NormalTok}[1]{#1}
\newcommand{\OperatorTok}[1]{\textcolor[rgb]{0.81,0.36,0.00}{\textbf{#1}}}
\newcommand{\OtherTok}[1]{\textcolor[rgb]{0.56,0.35,0.01}{#1}}
\newcommand{\PreprocessorTok}[1]{\textcolor[rgb]{0.56,0.35,0.01}{\textit{#1}}}
\newcommand{\RegionMarkerTok}[1]{#1}
\newcommand{\SpecialCharTok}[1]{\textcolor[rgb]{0.00,0.00,0.00}{#1}}
\newcommand{\SpecialStringTok}[1]{\textcolor[rgb]{0.31,0.60,0.02}{#1}}
\newcommand{\StringTok}[1]{\textcolor[rgb]{0.31,0.60,0.02}{#1}}
\newcommand{\VariableTok}[1]{\textcolor[rgb]{0.00,0.00,0.00}{#1}}
\newcommand{\VerbatimStringTok}[1]{\textcolor[rgb]{0.31,0.60,0.02}{#1}}
\newcommand{\WarningTok}[1]{\textcolor[rgb]{0.56,0.35,0.01}{\textbf{\textit{#1}}}}
\usepackage{graphicx,grffile}
\makeatletter
\def\maxwidth{\ifdim\Gin@nat@width>\linewidth\linewidth\else\Gin@nat@width\fi}
\def\maxheight{\ifdim\Gin@nat@height>\textheight\textheight\else\Gin@nat@height\fi}
\makeatother
% Scale images if necessary, so that they will not overflow the page
% margins by default, and it is still possible to overwrite the defaults
% using explicit options in \includegraphics[width, height, ...]{}
\setkeys{Gin}{width=\maxwidth,height=\maxheight,keepaspectratio}
\IfFileExists{parskip.sty}{%
\usepackage{parskip}
}{% else
\setlength{\parindent}{0pt}
\setlength{\parskip}{6pt plus 2pt minus 1pt}
}
\setlength{\emergencystretch}{3em}  % prevent overfull lines
\providecommand{\tightlist}{%
  \setlength{\itemsep}{0pt}\setlength{\parskip}{0pt}}
\setcounter{secnumdepth}{0}
% Redefines (sub)paragraphs to behave more like sections
\ifx\paragraph\undefined\else
\let\oldparagraph\paragraph
\renewcommand{\paragraph}[1]{\oldparagraph{#1}\mbox{}}
\fi
\ifx\subparagraph\undefined\else
\let\oldsubparagraph\subparagraph
\renewcommand{\subparagraph}[1]{\oldsubparagraph{#1}\mbox{}}
\fi

%%% Use protect on footnotes to avoid problems with footnotes in titles
\let\rmarkdownfootnote\footnote%
\def\footnote{\protect\rmarkdownfootnote}

%%% Change title format to be more compact
\usepackage{titling}

% Create subtitle command for use in maketitle
\providecommand{\subtitle}[1]{
  \posttitle{
    \begin{center}\large#1\end{center}
    }
}

\setlength{\droptitle}{-2em}

  \title{Chapter 2: Mixed-Effects Models for Factorial Designs}
    \pretitle{\vspace{\droptitle}\centering\huge}
  \posttitle{\par}
    \author{Keith Lohse, PhD, PStat}
    \preauthor{\centering\large\emph}
  \postauthor{\par}
      \predate{\centering\large\emph}
  \postdate{\par}
    \date{2/2/2020}


\begin{document}
\maketitle

For these examples, we are going to work with a fictional data that that
has two fully nested (i.e., between-subjects) and two fully crossed
(i.e., within-subject) factors. We have two hypothetical groups of older
adults (OA, \textgreater65 y) and younger adults (YA, \textless35 y).
Within each of those groups, half of the participants were in the
control group (C) and half were in the treatment group (T). Each
participant was also measured at three different times (1 to 3) in three
different conditions (A, B, and C). The first ten rows of these data can
be seen below.

\begin{Shaded}
\begin{Highlighting}[]
\KeywordTok{library}\NormalTok{(tidyverse); }\KeywordTok{library}\NormalTok{(RCurl); }\KeywordTok{library}\NormalTok{(ez); }\KeywordTok{library}\NormalTok{(lme4); }\KeywordTok{library}\NormalTok{(lmerTest)}

\NormalTok{DATA <{-}}\StringTok{ }\KeywordTok{read.csv}\NormalTok{(}\StringTok{"https://raw.githubusercontent.com/keithlohse/mixed\_effects\_models/master/data\_example.csv"}\NormalTok{)}
\end{Highlighting}
\end{Shaded}

\begin{Shaded}
\begin{Highlighting}[]
\KeywordTok{head}\NormalTok{(DATA, }\DecValTok{10}\NormalTok{)}
\end{Highlighting}
\end{Shaded}

\begin{verbatim}
##    subID group age_group condition time     speed
## 1     s1    CG        OA         A    1 0.6973960
## 2     s1    CG        OA         A    2 0.7605492
## 3     s1    CG        OA         A    3 0.6110076
## 4     s1    CG        OA         A    4 0.6156949
## 5     s1    CG        OA         B    1 1.2853857
## 6     s1    CG        OA         B    2 0.9632404
## 7     s1    CG        OA         B    3 0.7748278
## 8     s1    CG        OA         B    4 0.7083246
## 9     s1    CG        OA         C    1 0.8016847
## 10    s1    CG        OA         C    2 0.6802459
\end{verbatim}

Because we will want to ignore each of the within-subject variables at
different times, we will want to average across trials to create a data
set with one observation per condition (data\_COND), and we will want to
average across conditions to create a data set with one observation at
each time (data\_TIME).

Averaging across time, here are the first ten rows of data\_COND.

\begin{Shaded}
\begin{Highlighting}[]
\NormalTok{data\_COND <{-}}\StringTok{ }\KeywordTok{aggregate}\NormalTok{(speed }\OperatorTok{\textasciitilde{}}\StringTok{ }\NormalTok{subID }\OperatorTok{+}\StringTok{ }\NormalTok{condition }\OperatorTok{+}\StringTok{ }\NormalTok{age\_group }\OperatorTok{+}\StringTok{ }\NormalTok{group, }\DataTypeTok{data=}\NormalTok{DATA, }\DataTypeTok{FUN=}\NormalTok{mean)}
\NormalTok{data\_COND <{-}}\StringTok{ }\NormalTok{data\_COND }\OperatorTok{\%>\%}\StringTok{ }\KeywordTok{arrange}\NormalTok{(subID, age\_group, group, condition)}
\KeywordTok{head}\NormalTok{(data\_COND, }\DecValTok{10}\NormalTok{)}
\end{Highlighting}
\end{Shaded}

\begin{verbatim}
##    subID condition age_group group     speed
## 1     s1         A        OA    CG 0.6711620
## 2     s1         B        OA    CG 0.9329446
## 3     s1         C        OA    CG 0.6419632
## 4    s10         A        OA    CG 0.6803930
## 5    s10         B        OA    CG 0.8952121
## 6    s10         C        OA    CG 0.6388253
## 7    s11         A        YA    CG 0.6654044
## 8    s11         B        YA    CG 0.8697005
## 9    s11         C        YA    CG 0.6274834
## 10   s12         A        YA    CG 0.3940999
\end{verbatim}

Averaging across condiitons, here are the first ten rows of data\_TIME.

\begin{Shaded}
\begin{Highlighting}[]
\NormalTok{data\_TIME <{-}}\StringTok{ }\KeywordTok{aggregate}\NormalTok{(speed }\OperatorTok{\textasciitilde{}}\StringTok{ }\NormalTok{subID }\OperatorTok{+}\StringTok{ }\NormalTok{time }\OperatorTok{+}\StringTok{ }\NormalTok{age\_group }\OperatorTok{+}\StringTok{ }\NormalTok{group, }\DataTypeTok{data=}\NormalTok{DATA, }\DataTypeTok{FUN=}\NormalTok{mean)}
\NormalTok{data\_TIME <{-}}\StringTok{ }\NormalTok{data\_TIME }\OperatorTok{\%>\%}\StringTok{ }\KeywordTok{arrange}\NormalTok{(subID, age\_group, group, time)}
\KeywordTok{head}\NormalTok{(data\_TIME)}
\end{Highlighting}
\end{Shaded}

\begin{verbatim}
##   subID time age_group group     speed
## 1    s1    1        OA    CG 0.9281555
## 2    s1    2        OA    CG 0.8013452
## 3    s1    3        OA    CG 0.6469040
## 4    s1    4        OA    CG 0.6183551
## 5   s10    1        OA    CG 0.8973438
## 6   s10    2        OA    CG 0.7818411
\end{verbatim}

\hypertarget{one-way-repeated-measures-anova}{%
\section{One-Way Repeated Measures
ANOVA}\label{one-way-repeated-measures-anova}}

\hypertarget{a-single-crossed-factor}{%
\subsection{(A Single Crossed Factor)}\label{a-single-crossed-factor}}

For this example, we will focus on only the effect of condition, so we
will use the data\_COND dataset to average across different trials.
First, let's plot the data to get a better sense of what our data are
saying.

\begin{Shaded}
\begin{Highlighting}[]
\KeywordTok{ggplot}\NormalTok{(data\_COND, }\KeywordTok{aes}\NormalTok{(}\DataTypeTok{x =}\NormalTok{ condition, }\DataTypeTok{y =}\NormalTok{ speed)) }\OperatorTok{+}
\StringTok{  }\KeywordTok{geom\_point}\NormalTok{(}\KeywordTok{aes}\NormalTok{(}\DataTypeTok{fill=}\NormalTok{condition), }\DataTypeTok{pch=}\DecValTok{21}\NormalTok{, }\DataTypeTok{size=}\DecValTok{2}\NormalTok{,}
             \DataTypeTok{position=}\KeywordTok{position\_jitter}\NormalTok{(}\DataTypeTok{w=}\FloatTok{0.2}\NormalTok{, }\DataTypeTok{h=}\DecValTok{0}\NormalTok{))}\OperatorTok{+}
\StringTok{  }\KeywordTok{geom\_boxplot}\NormalTok{(}\KeywordTok{aes}\NormalTok{(}\DataTypeTok{fill=}\NormalTok{condition), }\DataTypeTok{col=}\StringTok{"black"}\NormalTok{, }
               \DataTypeTok{alpha=}\FloatTok{0.4}\NormalTok{, }\DataTypeTok{width=}\FloatTok{0.5}\NormalTok{)}\OperatorTok{+}
\StringTok{  }\KeywordTok{scale\_x\_discrete}\NormalTok{(}\DataTypeTok{name =} \StringTok{"Condition"}\NormalTok{) }\OperatorTok{+}
\StringTok{  }\KeywordTok{scale\_y\_continuous}\NormalTok{(}\DataTypeTok{name =} \StringTok{"Speed (m/s)"}\NormalTok{, }\DataTypeTok{limits =} \KeywordTok{c}\NormalTok{(}\DecValTok{0}\NormalTok{,}\DecValTok{3}\NormalTok{)) }\OperatorTok{+}
\StringTok{  }\KeywordTok{theme}\NormalTok{(}\DataTypeTok{axis.text=}\KeywordTok{element\_text}\NormalTok{(}\DataTypeTok{size=}\DecValTok{16}\NormalTok{, }\DataTypeTok{color=}\StringTok{"black"}\NormalTok{), }
        \DataTypeTok{axis.title=}\KeywordTok{element\_text}\NormalTok{(}\DataTypeTok{size=}\DecValTok{16}\NormalTok{, }\DataTypeTok{face=}\StringTok{"bold"}\NormalTok{),}
        \DataTypeTok{plot.title=}\KeywordTok{element\_text}\NormalTok{(}\DataTypeTok{size=}\DecValTok{16}\NormalTok{, }\DataTypeTok{face=}\StringTok{"bold"}\NormalTok{, }\DataTypeTok{hjust=}\FloatTok{0.5}\NormalTok{),}
        \DataTypeTok{panel.grid.minor =} \KeywordTok{element\_blank}\NormalTok{(),}
        \DataTypeTok{strip.text =} \KeywordTok{element\_text}\NormalTok{(}\DataTypeTok{size=}\DecValTok{16}\NormalTok{, }\DataTypeTok{face=}\StringTok{"bold"}\NormalTok{),}
        \DataTypeTok{legend.position =} \StringTok{"none"}\NormalTok{)}
\end{Highlighting}
\end{Shaded}

\includegraphics{lohse_MER_chapter_02_files/figure-latex/plotting the effects of condition-1.pdf}

\hypertarget{as-an-anova}{%
\subsection{As an ANOVA\ldots{}}\label{as-an-anova}}

To implement a simple one-way repeated measures ANOVA, we have a few
options. We could directly code our ANOVA using the aov() function in R:

\begin{Shaded}
\begin{Highlighting}[]
\KeywordTok{summary}\NormalTok{(}\KeywordTok{aov}\NormalTok{(speed }\OperatorTok{\textasciitilde{}}\StringTok{ }\NormalTok{condition }\OperatorTok{+}\StringTok{ }\KeywordTok{Error}\NormalTok{(subID}\OperatorTok{/}\NormalTok{condition), }\DataTypeTok{data=}\NormalTok{data\_COND))}
\end{Highlighting}
\end{Shaded}

\begin{verbatim}
## 
## Error: subID
##           Df Sum Sq Mean Sq F value Pr(>F)
## Residuals 39  3.374 0.08653               
## 
## Error: subID:condition
##           Df Sum Sq Mean Sq F value Pr(>F)    
## condition  2  1.554  0.7769    1595 <2e-16 ***
## Residuals 78  0.038  0.0005                   
## ---
## Signif. codes:  0 '***' 0.001 '**' 0.01 '*' 0.05 '.' 0.1 ' ' 1
\end{verbatim}

Or we can use the ezANOVA() function from the ``ez'' package.

\begin{Shaded}
\begin{Highlighting}[]
\KeywordTok{ezANOVA}\NormalTok{(}\DataTypeTok{data =}\NormalTok{ data\_COND, }
    \DataTypeTok{dv =}\NormalTok{ .(speed),}
    \DataTypeTok{wid =}\NormalTok{ .(subID),}
    \DataTypeTok{within =}\NormalTok{ .(condition)}
\NormalTok{)}
\end{Highlighting}
\end{Shaded}

\begin{verbatim}
## $ANOVA
##      Effect DFn DFd        F           p p<.05       ges
## 2 condition   2  78 1595.374 5.38041e-64     * 0.3128589
## 
## $`Mauchly's Test for Sphericity`
##      Effect         W         p p<.05
## 2 condition 0.9204751 0.2071234      
## 
## $`Sphericity Corrections`
##      Effect       GGe        p[GG] p[GG]<.05       HFe        p[HF] p[HF]<.05
## 2 condition 0.9263335 1.806101e-59         * 0.9705498 3.467917e-62         *
\end{verbatim}

Although these two different functions present the results in a slightly
different ways, note that the f-values for the two different omnibus
tests match. By either method, the f-observed for the main-effect of
condition is F(2,158) = 68.86, p \textless{} 0.001.

If you are familiar with issues of contrast versus treatment coding,
Type I versus Type III Sums of Squared Errors, and colinearity, then
directly controlling your own ANOVA using base R is probably a safe bet.
If you are not familiar with those terms/issues, then the ezANOVA code
is probably the best solution for a quick fix, but I would strongly
encourage you to develop a more detailed understanding of general-linear
models before jumping into mixed-effect models.

Specifically, by default the aov() function provides a test of your
statistical model using Type I Sums of Squared Errors, whereas the
ezANOVA() function provides a test of your statistical model using Type
III Sums of Squared Errors. In a completely orthogonal design where all
factors are statistically independent of each other (i.e., there is no
colinearity), the Type I and Type III Sums of Squared Errors will agree.
By default, R also used treatment coding (or ``dummy'' coding) for
categorical factors rather than orthogonal constrast codes. The omnibus
F-tests that we see in the ANOVA output will be the same regardless of
the types of codes used, but if we dig into individual regression
coefficients, it is important to remember how these variables were coded
so that we can interpret them correctly.

\hypertarget{one-way-rm-anova-as-a-mixed-effect-model}{%
\subsection{One-way RM ANOVA as a mixed-effect
model\ldots{}}\label{one-way-rm-anova-as-a-mixed-effect-model}}

Because we only have a single within-subject factor, we will only need
to add a random-effect of subject to account for individual differences
between subjects. By partitioning the between-subjects variance out of
our model, we can fairly test the effect of condition, because our
residuals will now be independent of each other.

\begin{Shaded}
\begin{Highlighting}[]
\CommentTok{\# First we will define our model}
\NormalTok{mod1 <{-}}\StringTok{ }\KeywordTok{lmer}\NormalTok{(speed }\OperatorTok{\textasciitilde{}}\StringTok{ }
\StringTok{               }\CommentTok{\# Fixed Effects:}
\StringTok{               }\NormalTok{condition }\OperatorTok{+}\StringTok{ }
\StringTok{               }\CommentTok{\# Random Effects: }
\StringTok{               }\NormalTok{(}\DecValTok{1}\OperatorTok{|}\NormalTok{subID), }
             \CommentTok{\# Define the data: }
             \DataTypeTok{data=}\NormalTok{data\_COND, }\DataTypeTok{REML =} \OtherTok{TRUE}\NormalTok{)}

\CommentTok{\# We can then get the ANOVA results for our model:}
\KeywordTok{anova}\NormalTok{(mod1)}
\end{Highlighting}
\end{Shaded}

\begin{verbatim}
## Type III Analysis of Variance Table with Satterthwaite's method
##           Sum Sq Mean Sq NumDF DenDF F value    Pr(>F)    
## condition 1.5537 0.77686     2    78  1595.4 < 2.2e-16 ***
## ---
## Signif. codes:  0 '***' 0.001 '**' 0.01 '*' 0.05 '.' 0.1 ' ' 1
\end{verbatim}

If we want to delve deeper into our model, we can also use the summary()
function to get more information about model fit statistics, parameter
estimates, random-effects and residuals.

\begin{Shaded}
\begin{Highlighting}[]
\KeywordTok{summary}\NormalTok{(mod1)}
\end{Highlighting}
\end{Shaded}

\begin{verbatim}
## Linear mixed model fit by REML. t-tests use Satterthwaite's method [
## lmerModLmerTest]
## Formula: speed ~ condition + (1 | subID)
##    Data: data_COND
## 
## REML criterion at convergence: -347.3
## 
## Scaled residuals: 
##      Min       1Q   Median       3Q      Max 
## -1.96018 -0.52737 -0.02761  0.51110  1.88647 
## 
## Random effects:
##  Groups   Name        Variance  Std.Dev.
##  subID    (Intercept) 0.0286797 0.16935 
##  Residual             0.0004869 0.02207 
## Number of obs: 120, groups:  subID, 40
## 
## Fixed effects:
##              Estimate Std. Error        df t value Pr(>|t|)    
## (Intercept)  0.630484   0.027003 39.879817  23.349   <2e-16 ***
## conditionB   0.239068   0.004934 78.000171  48.451   <2e-16 ***
## conditionC  -0.004558   0.004934 78.000171  -0.924    0.358    
## ---
## Signif. codes:  0 '***' 0.001 '**' 0.01 '*' 0.05 '.' 0.1 ' ' 1
## 
## Correlation of Fixed Effects:
##            (Intr) cndtnB
## conditionB -0.091       
## conditionC -0.091  0.500
\end{verbatim}

\hypertarget{two-way-repeated-measures-anova}{%
\section{Two-Way Repeated Measures
ANOVA}\label{two-way-repeated-measures-anova}}

\hypertarget{multiple-crossed-factors}{%
\subsection{(Multiple Crossed Factors)}\label{multiple-crossed-factors}}

For this example, we will be analyzing both Time and Condition, so we
will use our original full data frame, DATA. First, let's plot the data
to get a better sense of what our data look like.

\begin{Shaded}
\begin{Highlighting}[]
\NormalTok{DATA}\OperatorTok{$}\NormalTok{time <{-}}\StringTok{ }\KeywordTok{factor}\NormalTok{(DATA}\OperatorTok{$}\NormalTok{time)}

\KeywordTok{ggplot}\NormalTok{(DATA, }\KeywordTok{aes}\NormalTok{(}\DataTypeTok{x =}\NormalTok{ time, }\DataTypeTok{y =}\NormalTok{ speed)) }\OperatorTok{+}
\StringTok{  }\KeywordTok{geom\_point}\NormalTok{(}\KeywordTok{aes}\NormalTok{(}\DataTypeTok{fill=}\NormalTok{condition), }\DataTypeTok{pch=}\DecValTok{21}\NormalTok{, }\DataTypeTok{size=}\DecValTok{2}\NormalTok{,}
             \DataTypeTok{position=}\KeywordTok{position\_jitterdodge}\NormalTok{(}\DataTypeTok{dodge.width =} \FloatTok{0.5}\NormalTok{))}\OperatorTok{+}
\StringTok{  }\KeywordTok{geom\_boxplot}\NormalTok{(}\KeywordTok{aes}\NormalTok{(}\DataTypeTok{fill=}\NormalTok{condition), }\DataTypeTok{col=}\StringTok{"black"}\NormalTok{, }
               \DataTypeTok{alpha=}\FloatTok{0.4}\NormalTok{, }\DataTypeTok{width=}\FloatTok{0.5}\NormalTok{)}\OperatorTok{+}
\StringTok{  }\KeywordTok{scale\_x\_discrete}\NormalTok{(}\DataTypeTok{name =} \StringTok{"Time"}\NormalTok{) }\OperatorTok{+}
\StringTok{  }\KeywordTok{scale\_y\_continuous}\NormalTok{(}\DataTypeTok{name =} \StringTok{"Speed (m/s)"}\NormalTok{, }\DataTypeTok{limits =} \KeywordTok{c}\NormalTok{(}\DecValTok{0}\NormalTok{,}\DecValTok{3}\NormalTok{)) }\OperatorTok{+}
\StringTok{  }\KeywordTok{theme}\NormalTok{(}\DataTypeTok{axis.text=}\KeywordTok{element\_text}\NormalTok{(}\DataTypeTok{size=}\DecValTok{16}\NormalTok{, }\DataTypeTok{color=}\StringTok{"black"}\NormalTok{), }
        \DataTypeTok{axis.title=}\KeywordTok{element\_text}\NormalTok{(}\DataTypeTok{size=}\DecValTok{16}\NormalTok{, }\DataTypeTok{face=}\StringTok{"bold"}\NormalTok{),}
        \DataTypeTok{plot.title=}\KeywordTok{element\_text}\NormalTok{(}\DataTypeTok{size=}\DecValTok{16}\NormalTok{, }\DataTypeTok{face=}\StringTok{"bold"}\NormalTok{, }\DataTypeTok{hjust=}\FloatTok{0.5}\NormalTok{),}
        \DataTypeTok{panel.grid.minor =} \KeywordTok{element\_blank}\NormalTok{(),}
        \DataTypeTok{strip.text =} \KeywordTok{element\_text}\NormalTok{(}\DataTypeTok{size=}\DecValTok{16}\NormalTok{, }\DataTypeTok{face=}\StringTok{"bold"}\NormalTok{),}
        \DataTypeTok{legend.position =} \StringTok{"right"}\NormalTok{)}
\end{Highlighting}
\end{Shaded}

\includegraphics{lohse_MER_chapter_02_files/figure-latex/plotting two repeated measures-1.pdf}

\hypertarget{as-an-anova-1}{%
\subsection{As an ANOVA\ldots{}}\label{as-an-anova-1}}

Before we run our models, we want to convert time to a factor so that
our model is treating time categorically rather than continuously. (In
later modules, we will discuss how to mix continuous and categorical
factors.) Additionally, remember that we are now using our larger data
set ``DATA'' rather than our aggregated data set ``data\_COND''. To
implement a two-way repeated measures ANOVA, we have the same options as
before. We can directly code our ANOVA using the aov() function in R:

\begin{Shaded}
\begin{Highlighting}[]
\NormalTok{DATA}\OperatorTok{$}\NormalTok{time <{-}}\StringTok{ }\KeywordTok{factor}\NormalTok{(DATA}\OperatorTok{$}\NormalTok{time)}
\KeywordTok{summary}\NormalTok{(}\KeywordTok{aov}\NormalTok{(speed }\OperatorTok{\textasciitilde{}}\StringTok{ }\NormalTok{condition}\OperatorTok{*}\NormalTok{time }\OperatorTok{+}\StringTok{ }\KeywordTok{Error}\NormalTok{(subID}\OperatorTok{/}\NormalTok{(condition}\OperatorTok{*}\NormalTok{time)), }\DataTypeTok{data=}\NormalTok{DATA))}
\end{Highlighting}
\end{Shaded}

\begin{verbatim}
## 
## Error: subID
##           Df Sum Sq Mean Sq F value Pr(>F)
## Residuals 39   13.5  0.3461               
## 
## Error: subID:condition
##           Df Sum Sq Mean Sq F value Pr(>F)    
## condition  2  6.215  3.1074    1595 <2e-16 ***
## Residuals 78  0.152  0.0019                   
## ---
## Signif. codes:  0 '***' 0.001 '**' 0.01 '*' 0.05 '.' 0.1 ' ' 1
## 
## Error: subID:time
##            Df Sum Sq Mean Sq F value Pr(>F)    
## time        3 16.249   5.416   231.1 <2e-16 ***
## Residuals 117  2.742   0.023                   
## ---
## Signif. codes:  0 '***' 0.001 '**' 0.01 '*' 0.05 '.' 0.1 ' ' 1
## 
## Error: subID:condition:time
##                 Df Sum Sq Mean Sq F value Pr(>F)    
## condition:time   6 1.7004 0.28340   206.3 <2e-16 ***
## Residuals      234 0.3214 0.00137                   
## ---
## Signif. codes:  0 '***' 0.001 '**' 0.01 '*' 0.05 '.' 0.1 ' ' 1
\end{verbatim}

Or we can use the ezANOVA() function from the ``ez'' package.

\begin{Shaded}
\begin{Highlighting}[]
\KeywordTok{ezANOVA}\NormalTok{(}\DataTypeTok{data =}\NormalTok{ DATA, }
    \DataTypeTok{dv =}\NormalTok{ .(speed),}
    \DataTypeTok{wid =}\NormalTok{ .(subID),}
    \DataTypeTok{within =}\NormalTok{ .(time, condition)}
\NormalTok{)}
\end{Highlighting}
\end{Shaded}

\begin{verbatim}
## $ANOVA
##           Effect DFn DFd         F            p p<.05        ges
## 2           time   3 117  231.1075 5.469444e-49     * 0.49295527
## 3      condition   2  78 1595.3738 5.380410e-64     * 0.27105729
## 4 time:condition   6 234  206.3205 1.791079e-90     * 0.09234525
## 
## $`Mauchly's Test for Sphericity`
##           Effect           W            p p<.05
## 2           time 0.005506229 2.631500e-40     *
## 3      condition 0.920475134 2.071234e-01      
## 4 time:condition 0.543351720 3.192589e-01      
## 
## $`Sphericity Corrections`
##           Effect       GGe        p[GG] p[GG]<.05       HFe        p[HF]
## 2           time 0.3535446 6.354576e-19         * 0.3551751 5.333292e-19
## 3      condition 0.9263335 1.806101e-59         * 0.9705498 3.467917e-62
## 4 time:condition 0.8542512 1.120585e-77         * 0.9988847 2.243961e-90
##   p[HF]<.05
## 2         *
## 3         *
## 4         *
\end{verbatim}

Again, regardless of the coding approach that you use, these two
different functions produce the same f-observed for the main-effects of
time, F(2,158) = 676.1, p \textless{} 0.001, and Condition, F(2, 158) =
68.9, p \textless{} 0.001.

\hypertarget{two-way-rm-anova-as-a-mixed-effect-model}{%
\subsection{Two-way RM ANOVA as a mixed-effect
model\ldots{}}\label{two-way-rm-anova-as-a-mixed-effect-model}}

Because we now have two crossed factors, we need to not only account for
the fact that we have multiple observations for each participant, but we
have multiple observations coming from other factors. For instance, for
the effect of Condition, we actually have condition effects at Time A,
Time B, and Time C. The converse is also true for the effect of Time, we
have effects of Time in Condition A, Condition B, and Condition C. Thus,
in order to appropriately account for the statistical dependencies in
our data, we need to add random-effects of ``time:subID'' and
``condition:subID'' to the model.

\begin{Shaded}
\begin{Highlighting}[]
\CommentTok{\# First we will define our model}
\NormalTok{mod1 <{-}}\StringTok{ }\KeywordTok{lmer}\NormalTok{(speed }\OperatorTok{\textasciitilde{}}\StringTok{ }
\StringTok{               }\CommentTok{\# Fixed Effects:}
\StringTok{               }\NormalTok{time}\OperatorTok{*}\NormalTok{condition }\OperatorTok{+}\StringTok{ }
\StringTok{               }\CommentTok{\# Random Effects}
\StringTok{               }\NormalTok{(}\DecValTok{1}\OperatorTok{|}\NormalTok{subID)}\OperatorTok{+}\StringTok{ }\NormalTok{(}\DecValTok{1}\OperatorTok{|}\NormalTok{time}\OperatorTok{:}\NormalTok{subID) }\OperatorTok{+}\StringTok{ }\NormalTok{(}\DecValTok{1}\OperatorTok{|}\NormalTok{condition}\OperatorTok{:}\NormalTok{subID), }
               \CommentTok{\# Define your data, }
             \DataTypeTok{data=}\NormalTok{DATA, }\DataTypeTok{REML=}\OtherTok{TRUE}\NormalTok{)}

\CommentTok{\# We can then get the ANOVA results for our model:}
\KeywordTok{anova}\NormalTok{(mod1)}
\end{Highlighting}
\end{Shaded}

\begin{verbatim}
## Type III Analysis of Variance Table with Satterthwaite's method
##                Sum Sq Mean Sq NumDF   DenDF F value    Pr(>F)    
## time           0.9524 0.31746     3 117.002  231.11 < 2.2e-16 ***
## condition      4.3831 2.19157     2  78.001 1595.47 < 2.2e-16 ***
## time:condition 1.7004 0.28340     6 233.993  206.32 < 2.2e-16 ***
## ---
## Signif. codes:  0 '***' 0.001 '**' 0.01 '*' 0.05 '.' 0.1 ' ' 1
\end{verbatim}

If we want to delve deeper into our model, we can also use the summary()
function to get more information about model fit statistics, parameter
estimates, random-effects and residuals.

\begin{Shaded}
\begin{Highlighting}[]
\KeywordTok{summary}\NormalTok{(mod1)}
\end{Highlighting}
\end{Shaded}

\begin{verbatim}
## Linear mixed model fit by REML. t-tests use Satterthwaite's method [
## lmerModLmerTest]
## Formula: speed ~ time * condition + (1 | subID) + (1 | time:subID) + (1 |  
##     condition:subID)
##    Data: DATA
## 
## REML criterion at convergence: -1137.1
## 
## Scaled residuals: 
##      Min       1Q   Median       3Q      Max 
## -2.54791 -0.51044 -0.00404  0.51601  2.23932 
## 
## Random effects:
##  Groups          Name        Variance  Std.Dev.
##  time:subID      (Intercept) 0.0073542 0.08576 
##  condition:subID (Intercept) 0.0001435 0.01198 
##  subID           (Intercept) 0.0268437 0.16384 
##  Residual                    0.0013736 0.03706 
## Number of obs: 480, groups:  time:subID, 160; condition:subID, 120; subID, 40
## 
## Fixed effects:
##                    Estimate Std. Error         df t value Pr(>|t|)    
## (Intercept)        0.855586   0.029881  58.962904  28.633   <2e-16 ***
## time2             -0.207771   0.020890 145.043253  -9.946   <2e-16 ***
## time3             -0.331861   0.020890 145.043253 -15.886   <2e-16 ***
## time4             -0.360775   0.020890 145.043253 -17.270   <2e-16 ***
## conditionB         0.424710   0.008710 303.848137  48.764   <2e-16 ***
## conditionC         0.003757   0.008710 303.848137   0.431   0.6665    
## time2:conditionB  -0.146037   0.011720 233.993458 -12.460   <2e-16 ***
## time3:conditionB  -0.267839   0.011720 233.993458 -22.853   <2e-16 ***
## time4:conditionB  -0.328690   0.011720 233.993458 -28.045   <2e-16 ***
## time2:conditionC  -0.022298   0.011720 233.993458  -1.903   0.0583 .  
## time3:conditionC  -0.013171   0.011720 233.993458  -1.124   0.2623    
## time4:conditionC   0.002208   0.011720 233.993458   0.188   0.8507    
## ---
## Signif. codes:  0 '***' 0.001 '**' 0.01 '*' 0.05 '.' 0.1 ' ' 1
## 
## Correlation of Fixed Effects:
##             (Intr) time2  time3  time4  cndtnB cndtnC tm2:cB tm3:cB tm4:cB
## time2       -0.350                                                        
## time3       -0.350  0.500                                                 
## time4       -0.350  0.500  0.500                                          
## conditionB  -0.146  0.189  0.189  0.189                                   
## conditionC  -0.146  0.189  0.189  0.189  0.500                            
## tim2:cndtnB  0.098 -0.281 -0.140 -0.140 -0.673 -0.336                     
## tim3:cndtnB  0.098 -0.140 -0.281 -0.140 -0.673 -0.336  0.500              
## tim4:cndtnB  0.098 -0.140 -0.140 -0.281 -0.673 -0.336  0.500  0.500       
## tim2:cndtnC  0.098 -0.281 -0.140 -0.140 -0.336 -0.673  0.500  0.250  0.250
## tim3:cndtnC  0.098 -0.140 -0.281 -0.140 -0.336 -0.673  0.250  0.500  0.250
## tim4:cndtnC  0.098 -0.140 -0.140 -0.281 -0.336 -0.673  0.250  0.250  0.500
##             tm2:cC tm3:cC
## time2                    
## time3                    
## time4                    
## conditionB               
## conditionC               
## tim2:cndtnB              
## tim3:cndtnB              
## tim4:cndtnB              
## tim2:cndtnC              
## tim3:cndtnC  0.500       
## tim4:cndtnC  0.500  0.500
\end{verbatim}

\hypertarget{mixed-factorial-anova-with-one-crossed-factor}{%
\section{Mixed-Factorial ANOVA with One Crossed
Factor}\label{mixed-factorial-anova-with-one-crossed-factor}}

\hypertarget{split-plot-design}{%
\subsection{(Split-Plot Design)}\label{split-plot-design}}

For this example, we will now consider both of our between-subject
factors of age and treatment, as well as the within-subject factor of
condition. Do to this, we will average over the different times (1, 2,
and 3) to get one observation in each condition. This design is
sometimes referred as a ``mixed-factorial'' design, because we have a
mix of between-subjects and within-subject factors. Depending on your
background, you might be more familiar with this as a ``split plot''
design. Split plot refers to the fact that some plots are assigned to
different levels of Factor A (e.g., Treatment versus Control), but
within each plot there is a second level of randomization to different
levels of Factor B (e.g., Conditions A, B, or C). Both terms describe
the same thing but differ in their unit of analysis. In pscyhology, a
single person is usually the experimental unit (hence ``within-subject''
variables) whereas in agronomy or biology a physical area might be the
unit of analysis (hence ``split-plot'' variables). The more general way
to talk about these terms is as nested- or crossed-factors. For fully
nested factors, an experimental unit is represented in only one of the
factors (e.g., a person can only be in the treatment group or the
control group). For fully crossed factors, an experimental unit is
represented at all levels of the factor (e.g., e.g., a person is tested
in conditions A, B, and C).

In our example, Treatment and Age-Group are nested factors, because each
person is represented at only one level of each factor. However,
Condition is a crossed factor, because each person is represented at all
levels of Condition. We can see this more clearly if we plot all of our
data.

\begin{Shaded}
\begin{Highlighting}[]
\KeywordTok{ggplot}\NormalTok{(data\_COND, }\KeywordTok{aes}\NormalTok{(}\DataTypeTok{x =}\NormalTok{ condition, }\DataTypeTok{y =}\NormalTok{ speed)) }\OperatorTok{+}
\StringTok{  }\KeywordTok{geom\_point}\NormalTok{(}\KeywordTok{aes}\NormalTok{(}\DataTypeTok{fill=}\NormalTok{group), }\DataTypeTok{pch=}\DecValTok{21}\NormalTok{, }\DataTypeTok{size=}\DecValTok{2}\NormalTok{,}
             \DataTypeTok{position=}\KeywordTok{position\_jitterdodge}\NormalTok{(}\DataTypeTok{dodge.width =} \FloatTok{0.5}\NormalTok{))}\OperatorTok{+}
\StringTok{  }\KeywordTok{geom\_boxplot}\NormalTok{(}\KeywordTok{aes}\NormalTok{(}\DataTypeTok{fill=}\NormalTok{group), }\DataTypeTok{col=}\StringTok{"black"}\NormalTok{,}
               \DataTypeTok{alpha=}\FloatTok{0.4}\NormalTok{, }\DataTypeTok{width=}\FloatTok{0.5}\NormalTok{)}\OperatorTok{+}
\StringTok{  }\KeywordTok{facet\_wrap}\NormalTok{(}\OperatorTok{\textasciitilde{}}\NormalTok{age\_group)}\OperatorTok{+}
\StringTok{  }\KeywordTok{scale\_x\_discrete}\NormalTok{(}\DataTypeTok{name =} \StringTok{"Condition"}\NormalTok{) }\OperatorTok{+}
\StringTok{  }\KeywordTok{scale\_y\_continuous}\NormalTok{(}\DataTypeTok{name =} \StringTok{"Speed (m/s)"}\NormalTok{) }\OperatorTok{+}
\StringTok{  }\KeywordTok{theme}\NormalTok{(}\DataTypeTok{axis.text=}\KeywordTok{element\_text}\NormalTok{(}\DataTypeTok{size=}\DecValTok{16}\NormalTok{, }\DataTypeTok{color=}\StringTok{"black"}\NormalTok{), }
        \DataTypeTok{axis.title=}\KeywordTok{element\_text}\NormalTok{(}\DataTypeTok{size=}\DecValTok{16}\NormalTok{, }\DataTypeTok{face=}\StringTok{"bold"}\NormalTok{),}
        \DataTypeTok{plot.title=}\KeywordTok{element\_text}\NormalTok{(}\DataTypeTok{size=}\DecValTok{16}\NormalTok{, }\DataTypeTok{face=}\StringTok{"bold"}\NormalTok{, }\DataTypeTok{hjust=}\FloatTok{0.5}\NormalTok{),}
        \DataTypeTok{panel.grid.minor =} \KeywordTok{element\_blank}\NormalTok{(),}
        \DataTypeTok{strip.text =} \KeywordTok{element\_text}\NormalTok{(}\DataTypeTok{size=}\DecValTok{16}\NormalTok{, }\DataTypeTok{face=}\StringTok{"bold"}\NormalTok{),}
        \DataTypeTok{legend.position =} \StringTok{"right"}\NormalTok{)}
\end{Highlighting}
\end{Shaded}

\includegraphics{lohse_MER_chapter_02_files/figure-latex/plotting crossed and nested factors-1.pdf}

\hypertarget{as-an-anova-2}{%
\subsection{As an ANOVA\ldots{}}\label{as-an-anova-2}}

For this mixed factorial ANOVA, we have one factor of Condition that
varies within subjects, but we have two factors (Age Group and Treatment
Group) that vary between subjects. As before, we can directly code this
into our analysis of variance using the aov() function or using the
ezANOVA() function from the ``ez'' package.

\begin{Shaded}
\begin{Highlighting}[]
\KeywordTok{summary}\NormalTok{(}\KeywordTok{aov}\NormalTok{(speed }\OperatorTok{\textasciitilde{}}\StringTok{ }\NormalTok{age\_group}\OperatorTok{*}\NormalTok{group}\OperatorTok{*}\NormalTok{condition }\OperatorTok{+}\StringTok{ }\KeywordTok{Error}\NormalTok{(subID}\OperatorTok{/}\NormalTok{condition), }\DataTypeTok{data=}\NormalTok{data\_COND))}
\end{Highlighting}
\end{Shaded}

\begin{verbatim}
## 
## Error: subID
##                 Df Sum Sq Mean Sq F value   Pr(>F)    
## age_group        1 0.7346  0.7346  20.639 6.01e-05 ***
## group            1 1.3577  1.3577  38.148 4.04e-07 ***
## age_group:group  1 0.0009  0.0009   0.026    0.872    
## Residuals       36 1.2813  0.0356                     
## ---
## Signif. codes:  0 '***' 0.001 '**' 0.01 '*' 0.05 '.' 0.1 ' ' 1
## 
## Error: subID:condition
##                           Df Sum Sq Mean Sq  F value Pr(>F)    
## condition                  2 1.5537  0.7769 1587.609 <2e-16 ***
## age_group:condition        2 0.0004  0.0002    0.398  0.673    
## group:condition            2 0.0008  0.0004    0.814  0.447    
## age_group:group:condition  2 0.0016  0.0008    1.599  0.209    
## Residuals                 72 0.0352  0.0005                    
## ---
## Signif. codes:  0 '***' 0.001 '**' 0.01 '*' 0.05 '.' 0.1 ' ' 1
\end{verbatim}

Or we can use the ezANOVA() function from the ``ez'' package.

\begin{Shaded}
\begin{Highlighting}[]
\KeywordTok{ezANOVA}\NormalTok{(}\DataTypeTok{data =}\NormalTok{ data\_COND, }
    \DataTypeTok{dv =}\NormalTok{ .(speed),}
    \DataTypeTok{wid =}\NormalTok{ .(subID),}
    \DataTypeTok{within =}\NormalTok{ .(condition),}
    \DataTypeTok{between =}\NormalTok{ .(group, age\_group)}
\NormalTok{)}
\end{Highlighting}
\end{Shaded}

\begin{verbatim}
## $ANOVA
##                      Effect DFn DFd            F            p p<.05
## 2                     group   1  36 3.814802e+01 4.041332e-07     *
## 3                 age_group   1  36 2.063892e+01 6.014982e-05     *
## 5                 condition   2  72 1.587609e+03 2.815489e-60     *
## 4           group:age_group   1  36 2.626142e-02 8.721697e-01      
## 6           group:condition   2  72 8.137508e-01 4.472265e-01      
## 7       age_group:condition   2  72 3.977692e-01 6.732842e-01      
## 8 group:age_group:condition   2  72 1.598674e+00 2.092564e-01      
##            ges
## 2 0.5077066014
## 3 0.3581354009
## 5 0.5413227501
## 4 0.0007094581
## 6 0.0006045528
## 7 0.0002956025
## 8 0.0011869970
## 
## $`Mauchly's Test for Sphericity`
##                      Effect         W        p p<.05
## 5                 condition 0.8780828 0.102771      
## 6           group:condition 0.8780828 0.102771      
## 7       age_group:condition 0.8780828 0.102771      
## 8 group:age_group:condition 0.8780828 0.102771      
## 
## $`Sphericity Corrections`
##                      Effect       GGe        p[GG] p[GG]<.05       HFe
## 5                 condition 0.8913313 5.318648e-54         * 0.9345923
## 6           group:condition 0.8913313 4.352008e-01           0.9345923
## 7       age_group:condition 0.8913313 6.497733e-01           0.9345923
## 8 group:age_group:condition 0.8913313 2.119281e-01           0.9345923
##          p[HF] p[HF]<.05
## 5 1.686511e-56         *
## 6 4.401678e-01          
## 7 6.594669e-01          
## 8 2.109178e-01
\end{verbatim}

The ezANOVA() function provides us with more detail, for instance
automatically providing Maulchy's Test for Sphericity and both the
Greenhouse-Geisser and Hyun-Feldt corrections in the event that the
sphericity assumption is violated. Most importantly, the outputs of
these two functions agree when we look at the f-statistics for all
main-effects and interactions when sphericity is assumed.

\hypertarget{mixed-factorial-anova-as-a-mixed-effect-model}{%
\subsection{Mixed Factorial ANOVA as a mixed-effect
model\ldots{}}\label{mixed-factorial-anova-as-a-mixed-effect-model}}

For this mixed-factorial design, we need to account for the fact that we
have multiple observations coming from each person, so we will add a
random-effect of ``subID''. After accounting for this statistical
dependence in our data, we can now fairly test the effects of Group, Age
Group, and Condition with residuals that are independent of each other.

\begin{Shaded}
\begin{Highlighting}[]
\CommentTok{\# First we will define our model}
\NormalTok{mod1 <{-}}\StringTok{ }\KeywordTok{lmer}\NormalTok{(speed }\OperatorTok{\textasciitilde{}}\StringTok{ }
\StringTok{               }\CommentTok{\# Fixed Effects:}
\StringTok{               }\NormalTok{group}\OperatorTok{*}\NormalTok{age\_group}\OperatorTok{*}\NormalTok{condition }\OperatorTok{+}\StringTok{ }
\StringTok{               }\CommentTok{\# Random Effects}
\StringTok{               }\NormalTok{(}\DecValTok{1}\OperatorTok{|}\NormalTok{subID), }
               \CommentTok{\# Define your data, }
             \DataTypeTok{data=}\NormalTok{data\_COND, }\DataTypeTok{REML=}\OtherTok{TRUE}\NormalTok{)}

\CommentTok{\# We can then get the ANOVA results for our model:}
\KeywordTok{anova}\NormalTok{(mod1)}
\end{Highlighting}
\end{Shaded}

\begin{verbatim}
## Type III Analysis of Variance Table with Satterthwaite's method
##                            Sum Sq Mean Sq NumDF DenDF   F value    Pr(>F)    
## group                     0.01867 0.01867     1    36   38.1480 4.041e-07 ***
## age_group                 0.01010 0.01010     1    36   20.6389 6.015e-05 ***
## condition                 1.55371 0.77686     2    72 1587.6094 < 2.2e-16 ***
## group:age_group           0.00001 0.00001     1    36    0.0263    0.8722    
## group:condition           0.00080 0.00040     2    72    0.8138    0.4472    
## age_group:condition       0.00039 0.00019     2    72    0.3978    0.6733    
## group:age_group:condition 0.00156 0.00078     2    72    1.5987    0.2093    
## ---
## Signif. codes:  0 '***' 0.001 '**' 0.01 '*' 0.05 '.' 0.1 ' ' 1
\end{verbatim}

If we want to delve deeper into our model, we can also use the summary()
function to get more information about model fit statistics, parameter
estimates, random-effects and residuals.

\begin{Shaded}
\begin{Highlighting}[]
\KeywordTok{summary}\NormalTok{(mod1)}
\end{Highlighting}
\end{Shaded}

\begin{verbatim}
## Linear mixed model fit by REML. t-tests use Satterthwaite's method [
## lmerModLmerTest]
## Formula: speed ~ group * age_group * condition + (1 | subID)
##    Data: data_COND
## 
## REML criterion at convergence: -334.8
## 
## Scaled residuals: 
##      Min       1Q   Median       3Q      Max 
## -1.84873 -0.52691 -0.03524  0.48658  1.82075 
## 
## Random effects:
##  Groups   Name        Variance  Std.Dev.
##  subID    (Intercept) 0.0117005 0.10817 
##  Residual             0.0004893 0.02212 
## Number of obs: 120, groups:  subID, 40
## 
## Fixed effects:
##                                Estimate Std. Error        df t value Pr(>|t|)
## (Intercept)                    0.817351   0.034914 37.992655  23.410  < 2e-16
## groupT                        -0.221613   0.049376 37.992655  -4.488 6.47e-05
## age_groupYA                   -0.154073   0.049376 37.992655  -3.120  0.00344
## conditionB                     0.244157   0.009893 72.000000  24.681  < 2e-16
## conditionC                    -0.008051   0.009893 72.000000  -0.814  0.41842
## groupT:age_groupYA             0.003907   0.069828 37.992655   0.056  0.95567
## groupT:conditionB             -0.001356   0.013990 72.000000  -0.097  0.92305
## groupT:conditionC              0.011238   0.013990 72.000000   0.803  0.42448
## age_groupYA:conditionB        -0.022532   0.013990 72.000000  -1.611  0.11166
## age_groupYA:conditionC        -0.001427   0.013990 72.000000  -0.102  0.91907
## groupT:age_groupYA:conditionB  0.027420   0.019785 72.000000   1.386  0.17007
## groupT:age_groupYA:conditionC -0.005651   0.019785 72.000000  -0.286  0.77599
##                                  
## (Intercept)                   ***
## groupT                        ***
## age_groupYA                   ** 
## conditionB                    ***
## conditionC                       
## groupT:age_groupYA               
## groupT:conditionB                
## groupT:conditionC                
## age_groupYA:conditionB           
## age_groupYA:conditionC           
## groupT:age_groupYA:conditionB    
## groupT:age_groupYA:conditionC    
## ---
## Signif. codes:  0 '***' 0.001 '**' 0.01 '*' 0.05 '.' 0.1 ' ' 1
## 
## Correlation of Fixed Effects:
##             (Intr) groupT ag_gYA cndtnB cndtnC grT:_YA grpT:B grpT:C a_YA:B
## groupT      -0.707                                                         
## age_groupYA -0.707  0.500                                                  
## conditionB  -0.142  0.100  0.100                                           
## conditionC  -0.142  0.100  0.100  0.500                                    
## grpT:g_grYA  0.500 -0.707 -0.707 -0.071 -0.071                             
## grpT:cndtnB  0.100 -0.142 -0.071 -0.707 -0.354  0.100                      
## grpT:cndtnC  0.100 -0.142 -0.071 -0.354 -0.707  0.100   0.500              
## ag_grpYA:cB  0.100 -0.071 -0.142 -0.707 -0.354  0.100   0.500  0.250       
## ag_grpYA:cC  0.100 -0.071 -0.142 -0.354 -0.707  0.100   0.250  0.500  0.500
## grpT:g_YA:B -0.071  0.100  0.100  0.500  0.250 -0.142  -0.707 -0.354 -0.707
## grpT:g_YA:C -0.071  0.100  0.100  0.250  0.500 -0.142  -0.354 -0.707 -0.354
##             a_YA:C gT:_YA:B
## groupT                     
## age_groupYA                
## conditionB                 
## conditionC                 
## grpT:g_grYA                
## grpT:cndtnB                
## grpT:cndtnC                
## ag_grpYA:cB                
## ag_grpYA:cC                
## grpT:g_YA:B -0.354         
## grpT:g_YA:C -0.707  0.500
\end{verbatim}

\hypertarget{mixed-factorial-anova-with-multiple-crossed-factors}{%
\section{Mixed-Factorial ANOVA with Multiple Crossed
Factors}\label{mixed-factorial-anova-with-multiple-crossed-factors}}

\hypertarget{multi-way-split-plot-design}{%
\subsection{(Multi-Way Split-Plot
Design)}\label{multi-way-split-plot-design}}

Next, we will consider the more complicated example of our fully
factorial design with nested factors of Group and Age-Group, and crossed
factors of Condition and Time.

\begin{Shaded}
\begin{Highlighting}[]
\NormalTok{DATA}\OperatorTok{$}\NormalTok{time <{-}}\StringTok{ }\KeywordTok{factor}\NormalTok{(DATA}\OperatorTok{$}\NormalTok{time)}

\KeywordTok{ggplot}\NormalTok{(DATA, }\KeywordTok{aes}\NormalTok{(}\DataTypeTok{x =}\NormalTok{ time, }\DataTypeTok{y =}\NormalTok{ speed)) }\OperatorTok{+}
\StringTok{  }\KeywordTok{geom\_point}\NormalTok{(}\KeywordTok{aes}\NormalTok{(}\DataTypeTok{fill=}\NormalTok{group), }\DataTypeTok{size=}\DecValTok{2}\NormalTok{, }\DataTypeTok{shape=}\DecValTok{21}\NormalTok{,}
             \DataTypeTok{position=}\KeywordTok{position\_jitterdodge}\NormalTok{(}\DataTypeTok{dodge.width =} \FloatTok{0.5}\NormalTok{))}\OperatorTok{+}
\StringTok{  }\KeywordTok{geom\_boxplot}\NormalTok{(}\KeywordTok{aes}\NormalTok{(}\DataTypeTok{fill=}\NormalTok{group), }\DataTypeTok{col=}\StringTok{"black"}\NormalTok{, }
               \DataTypeTok{alpha=}\FloatTok{0.4}\NormalTok{, }\DataTypeTok{width=}\FloatTok{0.5}\NormalTok{)}\OperatorTok{+}
\StringTok{  }\KeywordTok{facet\_wrap}\NormalTok{(}\OperatorTok{\textasciitilde{}}\NormalTok{condition}\OperatorTok{+}\NormalTok{age\_group)}\OperatorTok{+}
\StringTok{  }\KeywordTok{scale\_x\_discrete}\NormalTok{(}\DataTypeTok{name =} \StringTok{"Time"}\NormalTok{) }\OperatorTok{+}
\StringTok{  }\KeywordTok{scale\_y\_continuous}\NormalTok{(}\DataTypeTok{name =} \StringTok{"Speed (m/s)"}\NormalTok{, }\DataTypeTok{limits =} \KeywordTok{c}\NormalTok{(}\DecValTok{0}\NormalTok{,}\DecValTok{3}\NormalTok{)) }\OperatorTok{+}
\StringTok{  }\KeywordTok{theme}\NormalTok{(}\DataTypeTok{axis.text=}\KeywordTok{element\_text}\NormalTok{(}\DataTypeTok{size=}\DecValTok{16}\NormalTok{, }\DataTypeTok{color=}\StringTok{"black"}\NormalTok{), }
        \DataTypeTok{axis.title=}\KeywordTok{element\_text}\NormalTok{(}\DataTypeTok{size=}\DecValTok{16}\NormalTok{, }\DataTypeTok{face=}\StringTok{"bold"}\NormalTok{),}
        \DataTypeTok{plot.title=}\KeywordTok{element\_text}\NormalTok{(}\DataTypeTok{size=}\DecValTok{16}\NormalTok{, }\DataTypeTok{face=}\StringTok{"bold"}\NormalTok{, }\DataTypeTok{hjust=}\FloatTok{0.5}\NormalTok{),}
        \DataTypeTok{panel.grid.minor =} \KeywordTok{element\_blank}\NormalTok{(),}
        \DataTypeTok{strip.text =} \KeywordTok{element\_text}\NormalTok{(}\DataTypeTok{size=}\DecValTok{16}\NormalTok{, }\DataTypeTok{face=}\StringTok{"bold"}\NormalTok{),}
        \DataTypeTok{legend.position =} \StringTok{"right"}\NormalTok{)}
\end{Highlighting}
\end{Shaded}

\includegraphics{lohse_MER_chapter_02_files/figure-latex/plotting multiple crossed and nested factors-1.pdf}

\hypertarget{as-an-anova-3}{%
\subsection{As an ANOVA\ldots{}}\label{as-an-anova-3}}

For this mixed factorial ANOVA, we have two factors that vary within
subjects, Condition and Time, and we have two factors that vary between
subjects, Age Group and treatment Group0. As before, we can directly
code this into our analysis of variance using the aov() function or
using the ezANOVA() function from the ``ez'' package.

\begin{Shaded}
\begin{Highlighting}[]
\KeywordTok{summary}\NormalTok{(}\KeywordTok{aov}\NormalTok{(speed }\OperatorTok{\textasciitilde{}}\StringTok{ }\NormalTok{age\_group}\OperatorTok{*}\NormalTok{group}\OperatorTok{*}\NormalTok{condition}\OperatorTok{*}\NormalTok{time }\OperatorTok{+}\StringTok{ }\KeywordTok{Error}\NormalTok{(subID}\OperatorTok{/}\NormalTok{(condition}\OperatorTok{*}\NormalTok{time)), }\DataTypeTok{data=}\NormalTok{DATA))}
\end{Highlighting}
\end{Shaded}

\begin{verbatim}
## 
## Error: subID
##                 Df Sum Sq Mean Sq F value   Pr(>F)    
## age_group        1  2.938   2.938  20.639 6.01e-05 ***
## group            1  5.431   5.431  38.148 4.04e-07 ***
## age_group:group  1  0.004   0.004   0.026    0.872    
## Residuals       36  5.125   0.142                     
## ---
## Signif. codes:  0 '***' 0.001 '**' 0.01 '*' 0.05 '.' 0.1 ' ' 1
## 
## Error: subID:condition
##                           Df Sum Sq Mean Sq  F value Pr(>F)    
## condition                  2  6.215  3.1074 1587.609 <2e-16 ***
## age_group:condition        2  0.002  0.0008    0.398  0.673    
## group:condition            2  0.003  0.0016    0.814  0.447    
## age_group:group:condition  2  0.006  0.0031    1.599  0.209    
## Residuals                 72  0.141  0.0020                    
## ---
## Signif. codes:  0 '***' 0.001 '**' 0.01 '*' 0.05 '.' 0.1 ' ' 1
## 
## Error: subID:time
##                       Df Sum Sq Mean Sq  F value Pr(>F)    
## time                   3 16.249   5.416 3955.080 <2e-16 ***
## age_group:time         3  2.585   0.862  629.291 <2e-16 ***
## group:time             3  0.002   0.001    0.491  0.689    
## age_group:group:time   3  0.007   0.002    1.649  0.182    
## Residuals            108  0.148   0.001                    
## ---
## Signif. codes:  0 '***' 0.001 '**' 0.01 '*' 0.05 '.' 0.1 ' ' 1
## 
## Error: subID:condition:time
##                                 Df Sum Sq Mean Sq F value  Pr(>F)    
## condition:time                   6 1.7004 0.28340 221.120 < 2e-16 ***
## age_group:condition:time         6 0.0107 0.00179   1.393 0.21858    
## group:condition:time             6 0.0241 0.00401   3.130 0.00581 ** 
## age_group:group:condition:time   6 0.0098 0.00163   1.274 0.27037    
## Residuals                      216 0.2768 0.00128                    
## ---
## Signif. codes:  0 '***' 0.001 '**' 0.01 '*' 0.05 '.' 0.1 ' ' 1
\end{verbatim}

\begin{Shaded}
\begin{Highlighting}[]
\KeywordTok{ezANOVA}\NormalTok{(}\DataTypeTok{data =}\NormalTok{ DATA, }
    \DataTypeTok{dv =}\NormalTok{ .(speed),}
    \DataTypeTok{wid =}\NormalTok{ .(subID),}
    \DataTypeTok{within =}\NormalTok{ .(condition, time),}
    \DataTypeTok{between =}\NormalTok{ .(group, age\_group)}
\NormalTok{)}
\end{Highlighting}
\end{Shaded}

\begin{verbatim}
## $ANOVA
##                            Effect DFn DFd            F             p p<.05
## 2                           group   1  36 3.814802e+01  4.041332e-07     *
## 3                       age_group   1  36 2.063892e+01  6.014982e-05     *
## 5                       condition   2  72 1.587609e+03  2.815489e-60     *
## 9                            time   3 108 3.955080e+03 3.170785e-110     *
## 4                 group:age_group   1  36 2.626142e-02  8.721697e-01      
## 6                 group:condition   2  72 8.137508e-01  4.472265e-01      
## 7             age_group:condition   2  72 3.977692e-01  6.732842e-01      
## 10                     group:time   3 108 4.906913e-01  6.894742e-01      
## 11                 age_group:time   3 108 6.292907e+02  3.217441e-68     *
## 13                 condition:time   6 216 2.211196e+02  2.717341e-89     *
## 8       group:age_group:condition   2  72 1.598674e+00  2.092564e-01      
## 12           group:age_group:time   3 108 1.648921e+00  1.824147e-01      
## 14           group:condition:time   6 216 3.130282e+00  5.810192e-03     *
## 15       age_group:condition:time   6 216 1.393005e+00  2.185833e-01      
## 16 group:age_group:condition:time   6 216 1.274136e+00  2.703707e-01      
##             ges
## 2  0.4883169213
## 3  0.3405069457
## 5  0.5220105678
## 9  0.7406180507
## 4  0.0006565407
## 6  0.0005594557
## 7  0.0002735456
## 10 0.0003541227
## 11 0.3123878323
## 13 0.2300612092
## 8  0.0010984999
## 12 0.0011890012
## 14 0.0042122123
## 15 0.0018788668
## 16 0.0017188131
## 
## $`Mauchly's Test for Sphericity`
##                            Effect         W          p p<.05
## 5                       condition 0.8780828 0.10277095      
## 6                 group:condition 0.8780828 0.10277095      
## 7             age_group:condition 0.8780828 0.10277095      
## 8       group:age_group:condition 0.8780828 0.10277095      
## 9                            time 0.7072129 0.03448091     *
## 10                     group:time 0.7072129 0.03448091     *
## 11                 age_group:time 0.7072129 0.03448091     *
## 12           group:age_group:time 0.7072129 0.03448091     *
## 13                 condition:time 0.6121757 0.68203573      
## 14           group:condition:time 0.6121757 0.68203573      
## 15       age_group:condition:time 0.6121757 0.68203573      
## 16 group:age_group:condition:time 0.6121757 0.68203573      
## 
## $`Sphericity Corrections`
##                            Effect       GGe        p[GG] p[GG]<.05       HFe
## 5                       condition 0.8913313 5.318648e-54         * 0.9345923
## 6                 group:condition 0.8913313 4.352008e-01           0.9345923
## 7             age_group:condition 0.8913313 6.497733e-01           0.9345923
## 8       group:age_group:condition 0.8913313 2.119281e-01           0.9345923
## 9                            time 0.8080836 1.443619e-89         * 0.8706410
## 10                     group:time 0.8080836 6.498993e-01           0.8706410
## 11                 age_group:time 0.8080836 1.282083e-55         * 0.8706410
## 12           group:age_group:time 0.8080836 1.924731e-01           0.8706410
## 13                 condition:time 0.8808334 5.733986e-79         * 1.0502199
## 14           group:condition:time 0.8808334 8.446041e-03         * 1.0502199
## 15       age_group:condition:time 0.8808334 2.259596e-01           1.0502199
## 16 group:age_group:condition:time 0.8808334 2.751532e-01           1.0502199
##           p[HF] p[HF]<.05
## 5  1.686511e-56         *
## 6  4.401678e-01          
## 7  6.594669e-01          
## 8  2.109178e-01          
## 9  2.659387e-96         *
## 10 6.637492e-01          
## 11 9.995687e-60         *
## 12 1.892283e-01          
## 13 2.717341e-89         *
## 14 5.810192e-03         *
## 15 2.185833e-01          
## 16 2.703707e-01
\end{verbatim}

As illustrated in the figure, these results show main-effects of\ldots{}

\hypertarget{multi-way-mixed-factorial-anova-as-a-mixed-effect-model}{%
\subsection{Multi-Way Mixed Factorial ANOVA as a mixed-effect
model\ldots{}}\label{multi-way-mixed-factorial-anova-as-a-mixed-effect-model}}

For this mixed-factorial design, we need to account for the fact that we
have multiple observations coming from each person, but we also need to
acocunt for the fact that we multiple observations for each Condition
and at each Time. In order to account for this non-independence in our
data, we need to include random-effects of subject, subject:condition,
and subject:time. Adding these random-effects to our model will make our
mixed-effects model statistically equivalent to the mixed-factorial
ANOVAs that we ran above.

\begin{Shaded}
\begin{Highlighting}[]
\CommentTok{\# First we will define our model}
\NormalTok{mod1 <{-}}\StringTok{ }\KeywordTok{lmer}\NormalTok{(speed }\OperatorTok{\textasciitilde{}}\StringTok{ }
\StringTok{               }\CommentTok{\# Fixed Effects:}
\StringTok{               }\NormalTok{group}\OperatorTok{*}\NormalTok{age\_group}\OperatorTok{*}\NormalTok{condition}\OperatorTok{*}\NormalTok{time }\OperatorTok{+}\StringTok{ }
\StringTok{               }\CommentTok{\# Random Effects}
\StringTok{               }\NormalTok{(}\DecValTok{1}\OperatorTok{|}\NormalTok{subID)}\OperatorTok{+}\NormalTok{(}\DecValTok{1}\OperatorTok{|}\NormalTok{condition}\OperatorTok{:}\NormalTok{subID)}\OperatorTok{+}\NormalTok{(}\DecValTok{1}\OperatorTok{|}\NormalTok{time}\OperatorTok{:}\NormalTok{subID), }
               \CommentTok{\# Define your data, }
             \DataTypeTok{data=}\NormalTok{DATA, }\DataTypeTok{REML=}\OtherTok{TRUE}\NormalTok{)}

\CommentTok{\# We can then get the ANOVA results for our model:}
\KeywordTok{anova}\NormalTok{(mod1)}
\end{Highlighting}
\end{Shaded}

\begin{verbatim}
## Type III Analysis of Variance Table with Satterthwaite's method
##                                 Sum Sq Mean Sq NumDF   DenDF   F value
## group                           0.0489  0.0489     1  36.006   38.1520
## age_group                       0.0265  0.0265     1  36.006   20.6411
## condition                       4.0703  2.0351     2  72.006 1587.8724
## time                           15.2054  5.0685     3 108.085 3954.5396
## group:age_group                 0.0000  0.0000     1  36.006    0.0263
## group:condition                 0.0021  0.0010     2  72.006    0.8139
## age_group:condition             0.0010  0.0005     2  72.006    0.3978
## group:time                      0.0019  0.0006     3 108.085    0.4906
## age_group:time                  2.4193  0.8064     3 108.085  629.2047
## condition:time                  1.7004  0.2834     6 216.095  221.1186
## group:age_group:condition       0.0041  0.0020     2  72.006    1.5989
## group:age_group:time            0.0063  0.0021     3 108.085    1.6487
## group:condition:time            0.0241  0.0040     6 216.095    3.1303
## age_group:condition:time        0.0107  0.0018     6 216.095    1.3930
## group:age_group:condition:time  0.0098  0.0016     6 216.095    1.2741
##                                   Pr(>F)    
## group                          4.035e-07 ***
## age_group                      6.009e-05 ***
## condition                      < 2.2e-16 ***
## time                           < 2.2e-16 ***
## group:age_group                  0.87216    
## group:condition                  0.44717    
## age_group:condition              0.67324    
## group:time                       0.68952    
## age_group:time                 < 2.2e-16 ***
## condition:time                 < 2.2e-16 ***
## group:age_group:condition        0.20920    
## group:age_group:time             0.18246    
## group:condition:time             0.00581 ** 
## age_group:condition:time         0.21858    
## group:age_group:condition:time   0.27037    
## ---
## Signif. codes:  0 '***' 0.001 '**' 0.01 '*' 0.05 '.' 0.1 ' ' 1
\end{verbatim}

If we want to delve deeper into our model, we can also use the summary()
function to get more information about model fit statistics, parameter
estimates, random-effects and residuals.

\begin{Shaded}
\begin{Highlighting}[]
\KeywordTok{summary}\NormalTok{(mod1)}
\end{Highlighting}
\end{Shaded}

\begin{verbatim}
## Linear mixed model fit by REML. t-tests use Satterthwaite's method [
## lmerModLmerTest]
## Formula: speed ~ group * age_group * condition * time + (1 | subID) +  
##     (1 | condition:subID) + (1 | time:subID)
##    Data: DATA
## 
## REML criterion at convergence: -1333.2
## 
## Scaled residuals: 
##      Min       1Q   Median       3Q      Max 
## -2.43548 -0.59557  0.01037  0.52619  2.51202 
## 
## Random effects:
##  Groups          Name        Variance  Std.Dev.
##  time:subID      (Intercept) 2.932e-05 0.005415
##  condition:subID (Intercept) 1.688e-04 0.012993
##  subID           (Intercept) 1.169e-02 0.108129
##  Residual                    1.282e-03 0.035801
## Number of obs: 480, groups:  time:subID, 160; condition:subID, 120; subID, 40
## 
## Fixed effects:
##                                       Estimate Std. Error         df t value
## (Intercept)                          8.956e-01  3.629e-02  4.434e+01  24.677
## groupT                              -1.935e-01  5.133e-02  4.434e+01  -3.769
## age_groupYA                          1.203e-01  5.133e-02  4.434e+01   2.344
## conditionB                           4.485e-01  1.703e-02  2.768e+02  26.333
## conditionC                           3.200e-02  1.703e-02  2.768e+02   1.879
## time2                               -4.860e-02  1.619e-02  3.237e+02  -3.002
## time3                               -1.224e-01  1.619e-02  3.237e+02  -7.557
## time4                               -1.420e-01  1.619e-02  3.237e+02  -8.769
## groupT:age_groupYA                  -1.367e-02  7.259e-02  4.434e+01  -0.188
## groupT:conditionB                   -8.582e-03  2.409e-02  2.768e+02  -0.356
## groupT:conditionC                   -2.316e-02  2.409e-02  2.768e+02  -0.962
## age_groupYA:conditionB              -4.019e-02  2.409e-02  2.768e+02  -1.669
## age_groupYA:conditionC              -3.111e-02  2.409e-02  2.768e+02  -1.291
## groupT:time2                        -2.301e-02  2.290e-02  3.237e+02  -1.005
## groupT:time3                        -3.381e-02  2.290e-02  3.237e+02  -1.477
## groupT:time4                        -5.576e-02  2.290e-02  3.237e+02  -2.435
## age_groupYA:time2                   -3.022e-01  2.290e-02  3.237e+02 -13.197
## age_groupYA:time3                   -3.926e-01  2.290e-02  3.237e+02 -17.144
## age_groupYA:time4                   -4.026e-01  2.290e-02  3.237e+02 -17.583
## conditionB:time2                    -1.732e-01  2.264e-02  2.161e+02  -7.647
## conditionC:time2                    -3.135e-02  2.264e-02  2.161e+02  -1.384
## conditionB:time3                    -2.734e-01  2.264e-02  2.161e+02 -12.074
## conditionC:time3                    -5.911e-02  2.264e-02  2.161e+02  -2.611
## conditionB:time4                    -3.709e-01  2.264e-02  2.161e+02 -16.379
## conditionC:time4                    -6.975e-02  2.264e-02  2.161e+02  -3.081
## groupT:age_groupYA:conditionB        2.358e-03  3.406e-02  2.768e+02   0.069
## groupT:age_groupYA:conditionC       -4.444e-03  3.406e-02  2.768e+02  -0.130
## groupT:age_groupYA:time2             1.376e-02  3.239e-02  3.237e+02   0.425
## groupT:age_groupYA:time3             1.484e-02  3.239e-02  3.237e+02   0.458
## groupT:age_groupYA:time4             4.169e-02  3.239e-02  3.237e+02   1.287
## groupT:conditionB:time2              6.052e-03  3.202e-02  2.161e+02   0.189
## groupT:conditionC:time2             -1.733e-02  3.202e-02  2.161e+02  -0.541
## groupT:conditionB:time3             -1.778e-02  3.202e-02  2.161e+02  -0.555
## groupT:conditionC:time3              6.519e-02  3.202e-02  2.161e+02   2.036
## groupT:conditionB:time4              4.063e-02  3.202e-02  2.161e+02   1.269
## groupT:conditionC:time4              8.973e-02  3.202e-02  2.161e+02   2.802
## age_groupYA:conditionB:time2         3.471e-02  3.202e-02  2.161e+02   1.084
## age_groupYA:conditionC:time2         1.141e-02  3.202e-02  2.161e+02   0.356
## age_groupYA:conditionB:time3        -1.168e-04  3.202e-02  2.161e+02  -0.004
## age_groupYA:conditionC:time3         3.155e-02  3.202e-02  2.161e+02   0.985
## age_groupYA:conditionB:time4         3.605e-02  3.202e-02  2.161e+02   1.126
## age_groupYA:conditionC:time4         7.577e-02  3.202e-02  2.161e+02   2.366
## groupT:age_groupYA:conditionB:time2  2.695e-02  4.528e-02  2.161e+02   0.595
## groupT:age_groupYA:conditionC:time2  4.804e-02  4.528e-02  2.161e+02   1.061
## groupT:age_groupYA:conditionB:time3  5.800e-02  4.528e-02  2.161e+02   1.281
## groupT:age_groupYA:conditionC:time3 -9.709e-03  4.528e-02  2.161e+02  -0.214
## groupT:age_groupYA:conditionB:time4  1.529e-02  4.528e-02  2.161e+02   0.338
## groupT:age_groupYA:conditionC:time4 -4.316e-02  4.528e-02  2.161e+02  -0.953
##                                     Pr(>|t|)    
## (Intercept)                          < 2e-16 ***
## groupT                               0.00048 ***
## age_groupYA                          0.02364 *  
## conditionB                           < 2e-16 ***
## conditionC                           0.06132 .  
## time2                                0.00290 ** 
## time3                               4.28e-13 ***
## time4                                < 2e-16 ***
## groupT:age_groupYA                   0.85153    
## groupT:conditionB                    0.72189    
## groupT:conditionC                    0.33714    
## age_groupYA:conditionB               0.09633 .  
## age_groupYA:conditionC               0.19764    
## groupT:time2                         0.31564    
## groupT:time3                         0.14075    
## groupT:time4                         0.01543 *  
## age_groupYA:time2                    < 2e-16 ***
## age_groupYA:time3                    < 2e-16 ***
## age_groupYA:time4                    < 2e-16 ***
## conditionB:time2                    6.66e-13 ***
## conditionC:time2                     0.16764    
## conditionB:time3                     < 2e-16 ***
## conditionC:time3                     0.00967 ** 
## conditionB:time4                     < 2e-16 ***
## conditionC:time4                     0.00233 ** 
## groupT:age_groupYA:conditionB        0.94486    
## groupT:age_groupYA:conditionC        0.89629    
## groupT:age_groupYA:time2             0.67115    
## groupT:age_groupYA:time3             0.64718    
## groupT:age_groupYA:time4             0.19889    
## groupT:conditionB:time2              0.85027    
## groupT:conditionC:time2              0.58897    
## groupT:conditionB:time3              0.57924    
## groupT:conditionC:time3              0.04300 *  
## groupT:conditionB:time4              0.20580    
## groupT:conditionC:time4              0.00553 ** 
## age_groupYA:conditionB:time2         0.27963    
## age_groupYA:conditionC:time2         0.72204    
## age_groupYA:conditionB:time3         0.99709    
## age_groupYA:conditionC:time3         0.32562    
## age_groupYA:conditionB:time4         0.26144    
## age_groupYA:conditionC:time4         0.01886 *  
## groupT:age_groupYA:conditionB:time2  0.55238    
## groupT:age_groupYA:conditionC:time2  0.28992    
## groupT:age_groupYA:conditionB:time3  0.20163    
## groupT:age_groupYA:conditionC:time3  0.83044    
## groupT:age_groupYA:conditionB:time4  0.73589    
## groupT:age_groupYA:conditionC:time4  0.34161    
## ---
## Signif. codes:  0 '***' 0.001 '**' 0.01 '*' 0.05 '.' 0.1 ' ' 1
\end{verbatim}

\begin{verbatim}
## 
## Correlation matrix not shown by default, as p = 48 > 12.
## Use print(x, correlation=TRUE)  or
##     vcov(x)        if you need it
\end{verbatim}


\end{document}
